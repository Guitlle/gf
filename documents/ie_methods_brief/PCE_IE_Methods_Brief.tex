%% BioMed_Central_Tex_Template_v1.06
%%                                      %
%  bmc_article.tex            ver: 1.06 %
%                                       %

%%IMPORTANT: do not delete the first line of this template
%%It must be present to enable the BMC Submission system to
%%recognise this template!!

%%%%%%%%%%%%%%%%%%%%%%%%%%%%%%%%%%%%%%%%%
%%                                     %%
%%  LaTeX template for BioMed Central  %%
%%     journal article submissions     %%
%%                                     %%
%%          <8 June 2012>              %%
%%                                     %%
%%                                     %%
%%%%%%%%%%%%%%%%%%%%%%%%%%%%%%%%%%%%%%%%%


%%%%%%%%%%%%%%%%%%%%%%%%%%%%%%%%%%%%%%%%%%%%%%%%%%%%%%%%%%%%%%%%%%%%%
%%                                                                 %%
%% For instructions on how to fill out this Tex template           %%
%% document please refer to Readme.html and the instructions for   %%
%% authors page on the biomed central website                      %%
%% http://www.biomedcentral.com/info/authors/                      %%
%%                                                                 %%
%% Please do not use \input{...} to include other tex files.       %%
%% Submit your LaTeX manuscript as one .tex document.              %%
%%                                                                 %%
%% All additional figures and files should be attached             %%
%% separately and not embedded in the \TeX\ document itself.       %%
%%                                                                 %%
%% BioMed Central currently use the MikTex distribution of         %%
%% TeX for Windows) of TeX and LaTeX.  This is available from      %%
%% http://www.miktex.org                                           %%
%%                                                                 %%
%%%%%%%%%%%%%%%%%%%%%%%%%%%%%%%%%%%%%%%%%%%%%%%%%%%%%%%%%%%%%%%%%%%%%

%%% additional documentclass options:
%  [doublespacing]
%  [linenumbers]   - put the line numbers on margins

%%% loading packages, author definitions

\documentclass[twocolumn]{bmcart}% uncomment this for twocolumn layout and comment line below
% \documentclass{bmcart}

%%% Load packages
%\usepackage{amsthm,amsmath}
\RequirePackage{natbib}
%\RequirePackage{hyperref}
\usepackage[utf8]{inputenc} %unicode support
%\usepackage[applemac]{inputenc} %applemac support if unicode package fails
%\usepackage[latin1]{inputenc} %UNIX support if unicode package fails
\renewcommand{\thefootnote}{\roman{footnote}}
\usepackage{wrapfig}
\usepackage{graphicx}
\usepackage{lipsum}

%%%%%%%%%%%%%%%%%%%%%%%%%%%%%%%%%%%%%%%%%%%%%%%%%
%%                                             %%
%%  If you wish to display your graphics for   %%
%%  your own use using includegraphic or       %%
%%  includegraphics, then comment out the      %%
%%  following two lines of code.               %%
%%  NB: These line *must* be included when     %%
%%  submitting to BMC.                         %%
%%  All figure files must be submitted as      %%
%%  separate graphics through the BMC          %%
%%  submission process, not included in the    %%
%%  submitted article.                         %%
%%                                             %%
%%%%%%%%%%%%%%%%%%%%%%%%%%%%%%%%%%%%%%%%%%%%%%%%%


% \def\includegraphic{}
% \def\includegraphics{}



%%% Put your definitions there:
\startlocaldefs
\endlocaldefs


%%% Begin ...
\begin{document}

%%% Start of article front matter
\begin{frontmatter}

\begin{fmbox}
\dochead{Methods Brief}

%%%%%%%%%%%%%%%%%%%%%%%%%%%%%%%%%%%%%%%%%%%%%%
%%                                          %%
%% Enter the title of your article here     %%
%%                                          %%
%%%%%%%%%%%%%%%%%%%%%%%%%%%%%%%%%%%%%%%%%%%%%%

\title{PCE Impact Evaluation Methodological Summary}

%%%%%%%%%%%%%%%%%%%%%%%%%%%%%%%%%%%%%%%%%%%%%%
%%                                          %%
%% Enter the authors here                   %%
%%                                          %%
%% Specify information, if available,       %%
%% in the form:                             %%
%%   <key>={<id1>,<id2>}                    %%
%%   <key>=                                 %%
%% Comment or delete the keys which are     %%
%% not used. Repeat \author command as much %%
%% as required.                             %%
%%                                          %%
%%%%%%%%%%%%%%%%%%%%%%%%%%%%%%%%%%%%%%%%%%%%%%

\author[
   addressref={aff1},                   % id's of addresses, e.g. {aff1,aff2}
   corref={aff1},                       % id of corresponding address, if any
   % noteref={n1},                        % id's of article notes, if any
   email={davidp6@uw.edu}   % email address
]{\inits{DEP}\fnm{David E} \snm{Phillips} \suffix{PhD}}

%%%%%%%%%%%%%%%%%%%%%%%%%%%%%%%%%%%%%%%%%%%%%%
%%                                          %%
%% Enter the authors' addresses here        %%
%%                                          %%
%% Repeat \address commands as much as      %%
%% required.                                %%
%%                                          %%
%%%%%%%%%%%%%%%%%%%%%%%%%%%%%%%%%%%%%%%%%%%%%%

\address[id=aff1]{%                           % unique id
  \orgname{Institute for Health Metrics and Evaluation}, % university, etc
  \street{University of Washington},                     %
  %\postcode{}                                % post or zip code
  \city{Seattle},                              % city
  \cny{USA}                                    % country
}

%%%%%%%%%%%%%%%%%%%%%%%%%%%%%%%%%%%%%%%%%%%%%%
%%                                          %%
%% Enter short notes here                   %%
%%                                          %%
%% Short notes will be after addresses      %%
%% on first page.                           %%
%%                                          %%
%%%%%%%%%%%%%%%%%%%%%%%%%%%%%%%%%%%%%%%%%%%%%%

\begin{artnotes}
%\note{Sample of title note}     % note to the article
% \note[id=n1]{Equal contributor} % note, connected to author
\end{artnotes}

\end{fmbox}% comment this for two column layout

%%%%%%%%%%%%%%%%%%%%%%%%%%%%%%%%%%%%%%%%%%%%%%
%%                                          %%
%% The Abstract begins here                 %%
%%                                          %%
%% Please refer to the Instructions for     %%
%% authors on http://www.biomedcentral.com  %%
%% and include the section headings         %%
%% accordingly for your article type.       %%
%%                                          %%
%%%%%%%%%%%%%%%%%%%%%%%%%%%%%%%%%%%%%%%%%%%%%%

\begin{abstractbox}

This is a draft and not intended for distribution. \\
\today

\end{abstractbox}
%
%\end{fmbox}% uncomment this for twcolumn layout

\end{frontmatter}

%%%%%%%%%%%%%%%%%%%%%%%%%%%%%%%%%%%%%%%%%%%%%%
%%                                          %%
%% The Main Body begins here                %%
%%                                          %%
%% Please refer to the instructions for     %%
%% authors on:                              %%
%% http://www.biomedcentral.com/info/authors%%
%% and include the section headings         %%
%% accordingly for your article type.       %%
%%                                          %%
%% See the Results and Discussion section   %%
%% for details on how to create sub-sections%%
%%                                          %%
%% use \cite{...} to cite references        %%
%%  \cite{koon} and                         %%
%%  \cite{oreg,khar,zvai,xjon,schn,pond}    %%
%%  \nocite{smith,marg,hunn,advi,koha,mouse}%%
%%                                          %%
%%%%%%%%%%%%%%%%%%%%%%%%%%%%%%%%%%%%%%%%%%%%%%

%%%%%%%%%%%%%%%%%%%%%%%%% start of article main body
% <put your article body there>

%%%%%%%%%%%%%%%%
%% Content    %%
%%
%
% To do:
% - use \neg instead of ! ?

% ------------------------------------------------------------------------------
% Overview
\section{Purpose of this Document}
The purpose of this document is twofold:

\begin{enumerate}
  \item To explain, in advance, the analytical approach which we have selected for impact evaluation, with the goals of methods improvement through feedback, transparency, and formal demonstration of the robustness (both strengths and limitations) of the methods. There are many aspects of this approach which I believe are as robust as possible, although the overall study design is weak.
  \item To lay out a "road map" of how we can accomplish a task as daunting as an impact evaluation of the Global Fund in 2 years. To detail, more precisely, how this can be done.
\end{enumerate}


% ------------------------------------------------------------------------------
% Overview
\section{Overview}
% it's mosty just measuring lots of indicators
The basic approach to this impact evaluation is simply to measure many separate indicators, then measure their correlation. Through impact evaluation frameworks (i.e. "results chains") and thematic frameworks, those indicators (and more) are outlined in specific detail. The remainder of the impact evaluation is to identify the appropriate indicators measure correlation between, and do so while controlling for confounding. \\

% majority of effort is to ensure that our data produce a valid measure of the quantity of interest
The majority of the complexity and effort required to accomplish this task is to ensure that the data we use actually produce a valid measure of the quantity of interest. Techniques such as data triangulation (see Section \ref{triangulation}), model-based geostatistics (see Section \ref{mbg}), outlier detection and multiple imputation (see Section \ref{data_processing}), will be applied selectively to enhance that measurement validity. \\

% controls are critical for the impact analysis. some are hard to come by like non-GF expenditure, some will have stronger correlation with GF expenditure than the output itself because endogeneity
The controls are especially critical for the impact analysis (see Section \ref{hypothesis}). Some other variables are expected to have a stronger correlation with both expenditure and outputs than the correlation between expenditure and outputs itself, and without careful attention to them, this analysis will fail to measure what it intends to measure.

% ------------------------------------------------------------------------------
% Background
% \section{Background}

% comment on study design



% ------------------------------------------------------------------------------
% Data Sources Overview
% \section{Data Sources Overview}



% ------------------------------------------------------------------------------
% Hypothesis Being Tested
\section{Hypothesis Being Tested} \label{hypothesis}

% that changes in global fund investments result in observable changes in outputs, and that those changes in outputs result in observable changes in coverage and, subsequently burden of disease
The core hypothesis of this impact evaluation is that changes in Global Fund investments result in observable changes in health systems outputs. Additionally, the hypothesis is that those changes in outputs result in observable changes in intervention coverage, which subsequently results in improvements in burden of disease. \\

% that conditional on need, Global Fund investments are exogenous
We assume that, conditional on need, Global Fund investments are exogenous to health systems outputs. In other words, we assume that investments are targeted towards interventions and areas with unmet need, and that apart from decision-making about where there is need for investment, health systems outputs are only affected by Global Fund investment and do not affect it. Need, in this context can be defined as burden of disease without proportionate spending by other financing sources. \\

% how confounding is expected to work

% ------------------------------------------------------------------------------
% Basic Outputs Model
\section{Basic Outputs Model}

% the form of the model
The central model for measuring the contribution of Global Fund inputs to health systems outputs is a linear model. The precise functional form of the model will be selected to be fit for purpose given the distribution of the output data. In theory, the characteristics of typical output data are that they are integer (whole numbers), over-disubrsed (requiring both a mean and variance parameter to describe their distribution), heteroskedastic (different variance at different levels of a given explanatory variable), and temporally-autocorrelated (similar to itself in previous time points). For these reasons, the the model is likely most appropriately fit as a generalized linear mixed model (GLMM) in the negative binomial family. For simplicity, I will avoid model notation that explictly represents a negative binomial GLMM, in order to focus attention on the variables and the level at which they're measured (indexed). \\

% principle of the model
Measurement of the Global Fund contribution to outputs will rely on the co-variance of outputs and expenditure along three dimensions: space, time and activity. In other words, investments by the Global Fund focus on different interventions in different places, and change from year to year, and so do outputs. If the hypothesis is correct, we would expect to observe changes in health systems outputs that coincide with changes in investments along those three dimensions. As noted already, that coincidence alone is not expected to reflect the Global Fund's contribution, but rather the coincidence that happened apart from the correlation that other resources and burden of disease have with outputs. That implies the following regression:

% the model
\begin{equation} \label{basic_model}
O_{jti} = \beta_1 E^{gf}_{jti} + \beta_2 E^{!gf}_{jti} + \beta_3 I_{jt}
\end{equation}

% explanation of notation
In this formula, $j$ indexes subnational areas (provinces/ departments, districts/municipalities or otherwise), $t$ indexes time (years, quarters, or months) and $i$ indexes intervention categories, defined by the Modular Framework. $O$, represents counts of health systems outputs, $E$ represents expenditure either by the Global Fund ($^gf$) or other sources ($^!gf$), and $I$ represents indicence. The $\beta$ terms are coefficients (correlations) to be estimated from the data. Most likely, this model will be fit separately by disease.\\

% interpretation
The coefficient $\beta_1$ therefore measures the correlation between Global Fund investment and outputs, controlling for other investments and disease burden. Under the assumptions of the Theory of Change and Impact Frameworks (detailed elsewhere), this correlation represents the contribution of the Global Fund to those outputs. \\

% extension to RSSH
\subsection{Extension to RSSH}
In evaluating the contribution of catalytic and system-wide Global Fund investments (such as investments to strengthen resilient and sustainable systems for health, abbreviated RSSH), an additional layer of controls may be necessary. This is because such investments are intended to operate in addition to, or synergistically with, other program areas, not to result in outputs on their own. Essentially, this ammounts to controlling for Global Fund spending from programs other than just RSSH:

% RSSH model
\begin{equation}
O_{jti} = \beta_1 E^{RSSH}_{jti} + \beta_1 E^{HIV/TB/Malaria}_{jti} + \beta_2 E^{!gf}_{jti} + \beta_3 I_{jt}
\end{equation}

% note about using budget as an ITT
\textit{Technical Note: Although expenditure from the Global Fund is the intervention of interest, the above model will also be explored using the planned budget as a proxy for expenditure. The rationale for doing so is to avoid potential bias from low absorption. Similar to the "intent to treat" principle from a randomized control trial, the drivers of actual execution of funds are more diverse and poorly-understood than the drivers of planned budgeting, and thus more challenging to control. To measure the correlation between actual expenditure and outputs would also be to measure the contribution of those absorption-driving factors, and would weaken the interpretability of the coefficient of interest. \\}

% ------------------------------------------------------------------------------
% Worked Example
\section{Worked Example}
This section provides an example of the model above, using simulated data.


% ------------------------------------------------------------------------------
% Further Details on Resource Tracking
\section{Further Details on Resource Tracking} \label{resource_tracking}

\subsection{Purely systematic Model}


\subsection{Stochastic, Cost-Based Model}


% ------------------------------------------------------------------------------
% Further Details on Triangulation
\section{Further Details on Data Processing} \label{data_processing}

% outliers

% missing data


% ------------------------------------------------------------------------------
% Further Details on Triangulation
\section{Further Details on Data Triangulation} \label{triangulation}



% ------------------------------------------------------------------------------
% Further Details on Model-Based Geostatistics
\section{Further Details on Model-Based Geostatistics} \label{mbg}



% ------------------------------------------------------------------------------
% Further Details on Model-Based Geostatistics
\section{Relationship with Value for Money} \label{vfm}
The basic outputs model has a natural relationship with value for money (VfM) assessment. As defined by the Global Fund Monitoring and Evaluation team, the key metric of VfM is cost per output, or cost per case averted. \cite{kpi_team} \\
% CITATION: Briefing
% Strategic Information Department Briefing
% 2017-2022 Strategic KPI Framework Performance targets for KPIs 1, 2 & 8
% March 2018. The Global Fund

The definition of the coefficient $\beta_1$ in formula \ref{basic_model} is the average observed increase in outputs per unit increase in expenditure. $\beta_1$ is also simply stated as the inverse of cost per output (i.e. $1/\beta_1$). In this way, the core model for impact evaluation is the same as the core model for VfM assessment. \\

Additional interpretation of the impact evaluation model is the key to using it for VfM assessment. As allocation efficiency is among the critical topics for evaluation, estimates from the core model can be produced among counterfactual scenarios, comparing alternate mixes of interventions and their expected relationship with output.\\

%%%%%%%%%%%%%%%%%%%%%%%%%%%%%%%%%%%%%%%%%%%%%%
%%                                          %%
%% Backmatter begins here                   %%
%%                                          %%
%%%%%%%%%%%%%%%%%%%%%%%%%%%%%%%%%%%%%%%%%%%%%%

\section{References}


\begin{backmatter}

%%%%%%%%%%%%%%%%%%%%%%%%%%%%%%%%%%%%%%%%%%%%%%%%%%%%%%%%%%%%%
%%                  The Bibliography                       %%
%%                                                         %%
%%  Bmc_mathpys.bst  will be used to                       %%
%%  create a .BBL file for submission.                     %%
%%  After submission of the .TEX file,                     %%
%%  you will be prompted to submit your .BBL file.         %%
%%                                                         %%
%%                                                         %%
%%  Note that the displayed Bibliography will not          %%
%%  necessarily be rendered by Latex exactly as specified  %%
%%  in the online Instructions for Authors.                %%
%%                                                         %%
%%%%%%%%%%%%%%%%%%%%%%%%%%%%%%%%%%%%%%%%%%%%%%%%%%%%%%%%%%%%%

% if your bibliography is in bibtex format, use those commands:
% \bibliographystyle{bmc-mathphys} % Style BST file
% \bibliographystyle{unsrt} % Style BST file
% \bibliography{VSPI_Work_Plan}      % Bibliography file (usually '*.bib' )

% or include bibliography directly:
% \begin{thebibliography}
% \bibitem{b1}
% \end{thebibliography}

%%%%%%%%%%%%%%%%%%%%%%%%%%%%%%%%%%%
%%                               %%
%% Figures                       %%
%%                               %%
%% NB: this is for captions and  %%
%% Titles. All graphics must be  %%
%% submitted separately and NOT  %%
%% included in the Tex document  %%
%%                               %%
%%%%%%%%%%%%%%%%%%%%%%%%%%%%%%%%%%%

%%
%% Do not use \listoffigures as most will included as separate files

% \section*{Figures}
  % \begin{figure}[h!]
  % \caption{\csentence{Sample figure title.}
      % A short description of the figure content
      % should go here.}
      % \end{figure}

% \begin{figure}[h!]
  % \caption{\csentence{Sample figure title.}
      % Figure legend text.}
      % \end{figure}

%%%%%%%%%%%%%%%%%%%%%%%%%%%%%%%%%%%
%%                               %%
%% Tables                        %%
%%                               %%
%%%%%%%%%%%%%%%%%%%%%%%%%%%%%%%%%%%

% Use of \listoftables is discouraged.

% \section*{Tables}
% \begin{table}[h!]
% \caption{Sample table title. This is where the description of the table should go.}
      % \begin{tabular}{cccc}
        % \hline
           % & B1  &B2   & B3\\ \hline
        % A1 & 0.1 & 0.2 & 0.3\\
        % A2 & ... & ..  & .\\
        % A3 & ..  & .   & .\\ \hline
      % \end{tabular}
% \end{table}

%%%%%%%%%%%%%%%%%%%%%%%%%%%%%%%%%%%
%%                               %%
%% Additional Files              %%
%%                               %%
%%%%%%%%%%%%%%%%%%%%%%%%%%%%%%%%%%%


\end{backmatter}
\end{document}
